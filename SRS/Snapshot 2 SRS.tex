\documentclass[15pt]{article}
\usepackage[a4paper, margin=1in]{geometry}
\usepackage{titlesec}
\usepackage{hyperref}
\usepackage{graphicx}
\usepackage{fancyhdr}
\usepackage{enumitem}

\pagestyle{fancy}
\fancyhf{}
\rhead{Software Requirements Specification}
\lhead{Titanic Space Ship AI Model}
\rfoot{\thepage}

\title{Software Requirements Specification (SRS)\\\large Spaceship Titanic AI Prediction System}
\author{Team Name or Author(s)}
\date{\today}

\begin{document}

\maketitle
\tableofcontents
\newpage

\section{Introduction}
\subsection{Purpose}
This document defines the functional and non-functional requirements for a machine learning–based system designed to predict whether a passenger aboard the Spaceship Titanic was transported to another dimension due to a spacetime anomaly.

\subsection{Scope}
The system will:
\begin{itemize}
  \item Ingest spaceship passenger data in CSV format.
  \item Preprocess data (handle missing values, encode categorical variables).
  \item Train a machine learning model using historical labeled outcomes.
  \item Generate predictions for new passenger records.
  \item Read runtime parameters (classification threshold, file paths, hyperparameters) from an external configuration file.
  \item Log errors and send alerts (e.g., Slack or email) on critical failures.
  \item Export predictions to CSV for downstream integration.
\end{itemize}

\subsection{Definitions, Acronyms, and Abbreviations}
\begin{table}[h!]
  \centering
  \begin{tabular}{|l|l|p{8cm}|}
  \hline
  \textbf{Acronym} & \textbf{Full Form}                       & \textbf{Description} \\
  \hline
  AI    & Artificial Intelligence                  & Systems that simulate human intelligence in machines. \\
  ML    & Machine Learning                         & A subset of AI that enables models to learn from data. \\
  CSV   & Comma-Separated Values                   & Common format for structured tabular data files. \\
  CFG   & Configuration File                       & A JSON or YAML file specifying runtime parameters. \\
  API   & Application Programming Interface        & A set of rules allowing software components to communicate. \\
  SRS   & Software Requirements Specification      & A document that describes what the system should do. \\
  \hline
  \end{tabular}
  \caption{Acronyms and Abbreviations}
\end{table}

\subsection{References}
\begin{itemize}
  \item Kaggle Competition: \url{https://www.kaggle.com/competitions/spaceship-titanic/overview}
  \item IEEE Standard for SRS: IEEE 830
  \item Python Library Docs: \texttt{pandas}, \texttt{scikit-learn}, \texttt{xgboost}
\end{itemize}

\subsection{Overview}
The remainder of this document specifies detailed functional and non-functional requirements, external interfaces, and acceptance criteria.

\section{Overall Description}
\subsection{Product Perspective}
This is a standalone AI prediction system that supports interstellar rescue operations by classifying which passengers were transported during a spacetime anomaly.

\subsection{Product Functions}
\begin{enumerate}[label=PF\arabic*:]
  \item Load passenger data from a CSV file.
  \item Preprocess data:
    \begin{itemize}
      \item Impute missing values.
      \item Encode categorical variables (e.g., one-hot encoding).
      \item Normalize or scale features if necessary.
    \end{itemize}
  \item Train a supervised classification model (e.g., XGBoost).
  \item Split data into training (80 \%) and validation (20 \%) sets.
  \item Evaluate model performance using metrics (accuracy, ROC-AUC).
  \item Read configuration parameters (threshold, file paths, hyperparameters) from an external JSON configuration file named \texttt{config.json}.
  \item Generate transport predictions for unseen data.
  \item Log all processing steps and errors; on critical failures send alerts via Slack or email.
  \item Export prediction results to a CSV file with columns: PassengerID, Probability, PredictedLabel.
\end{enumerate}

\subsection{User Classes and Characteristics}
\begin{itemize}
  \item \textbf{Data Scientists:} Configure and maintain model training pipeline.
  \item \textbf{Rescue Analysts:} Use exported predictions to plan rescue missions.
  \item \textbf{Developers:} Extend and deploy system; add new features.
\end{itemize}

\subsection{Operating Environment}
\begin{itemize}
  \item Google Colab or equivalent Jupyter environment.
  \item Python 3.8+ runtime.
  \item Internet access for cloud-based environments.
\end{itemize}

\subsection{Design and Implementation Constraints}
\begin{itemize}
  \item Limited GPU/TPU session time on free Colab.
  \item Compliance with interstellar data privacy regulations.
  \item PEP 8 coding standards enforced.
\end{itemize}

\subsection{User Documentation}
\begin{itemize}
  \item README.md with setup instructions.
  \item User guide describing how to edit \texttt{config.json}, run training, and execute predictions.
\end{itemize}

\subsection{Assumptions and Dependencies}
\begin{itemize}
  \item Input CSV files conform to the specified schema.
  \item \texttt{config.json} is correctly formatted and located in the working directory.
  \item Email/Slack credentials are configured for alerting.
\end{itemize}

\section{Specific Requirements}
\subsection{Functional Requirements}
\begin{enumerate}[label=FR\arabic*:]
  \item System shall load a CSV file containing passenger records (FR1).
  \item System shall preprocess data: impute missing values, encode categorical features (FR2).
  \item System shall split data into 80 \% training and 20 \% validation sets (FR3).
  \item System shall train a classification model and save the model artifact (FR4).
  \item System shall read runtime parameters from \texttt{config.json}, including classification threshold, file paths, hyperparameters (FR5).
  \item System shall generate transport probability and label for new passenger data (FR6).
  \item System shall export predictions to a CSV file (FR7).
  \item System shall log all processing steps and errors; send alerts for critical failures (FR8).
\end{enumerate}

\subsection{Non-Functional Requirements}
\begin{enumerate}[label=NFR\arabic*:]
  \item Predictions for datasets under 1,000 records must complete within 60 seconds (NFR1).
  \item Model validation AUC must exceed 0.80 on the validation set (NFR2).
  \item Sensitive passenger data must be anonymized during processing (NFR3).
  \item All code shall conform to PEP 8 style guidelines (NFR4).
  \item Critical errors shall be logged and alerts sent within 5 minutes of detection (NFR5).
\end{enumerate}

\subsection{External Interface Requirements}
\begin{itemize}
  \item \textbf{User Interface:} Command‐line interface to start training and prediction jobs.
  \item \textbf{Software Interfaces:}  
    \begin{itemize}
      \item Python 3.x libraries: \texttt{pandas}, \texttt{scikit-learn}, \texttt{xgboost}, \texttt{json}, \texttt{logging}.
      \item Slack or SMTP for alert notifications.
    \end{itemize}
\end{itemize}

\section{Appendices}
\subsection{Appendix A: Future Enhancements}
\begin{itemize}
  \item Docker-compose definitions for full stack deployment.
  \item Integration of model explainability (e.g., SHAP).
\end{itemize}

\subsection{Appendix B: Revision History}
\begin{tabular}{|l|l|l|l|}
\hline
Version & Date & Author & Description \\
\hline
1.0 & \today & Gauri Chahal & Initial SRS draft \\
\hline
1.1 & \today & Gauri Chahal & Added configuration management and logging requirements \\
\hline
\end{tabular}

\end{document}
