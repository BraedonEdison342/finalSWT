\documentclass[15pt]{article}
\usepackage[a4paper, margin=1in]{geometry}
\usepackage{titlesec}
\usepackage{hyperref}
\usepackage{graphicx}
\usepackage{fancyhdr}
\usepackage{enumitem}

\pagestyle{fancy}
\fancyhf{}
\rhead{Software Requirements Specification}
\lhead{Titanic Space Ship AI Model}
\rfoot{\thepage}

\title{Software Requirements Specification (SRS)\\\large Spaceship Titanic AI Prediction System}
\author{Team Name or Author(s)}
\date{\today}

\begin{document}

\maketitle
\tableofcontents
\newpage

\section{Introduction}
\subsection{Purpose}
This document defines the functional and non-functional requirements for a machine learning-based system designed to predict whether a passenger aboard the Spaceship Titanic was transported to another dimension due to a spacetime anomaly.

\subsection{Scope}
The system will ingest spaceship passenger data, preprocess it, train a model using historical outcomes, and generate predictions for new data. Predictions are aimed at supporting interstellar rescue teams.

\subsection{Definitions, Acronyms, and Abbreviations}
Use the same acronym table from the SDD or include a trimmed version relevant to requirements only.

\subsection{References}
\begin{itemize}
    \item Kaggle Competition: \url{https://www.kaggle.com/competitions/spaceship-titanic/overview}
    \item IEEE Standard for SRS: IEEE 830
    \item Python Library Docs: \texttt{pandas}, \texttt{scikit-learn}, \texttt{xgboost}
\end{itemize}

\subsection{Overview}
The rest of the document outlines detailed requirements including system features, interfaces, constraints, and acceptance criteria.

\section{Overall Description}
\subsection{Product Perspective}
This is a standalone prediction system utilizing machine learning to support interstellar anomaly investigations.

\subsection{Product Functions}
\begin{itemize}
    \item Load passenger data (CSV).
    \item Preprocess data: handle missing values, encode categorical variables.
    \item Train ML model using labeled data.
    \item Predict transported status for new passengers.
    \item Display or export results.
\end{itemize}

\subsection{User Classes and Characteristics}
\begin{itemize}
    \item \textbf{Data Scientists:} Maintain the model pipeline.
    \item \textbf{Rescue Analysts:} Use predictions to prioritize recovery.
    \item \textbf{Developers:} Update system as new features or models are added.
\end{itemize}

\subsection{Operating Environment}
\begin{itemize}
    \item Google Colab / Jupyter Notebook
    \item Python 3.x
    \item Compatible browser for UI (Chrome, Firefox)
\end{itemize}

\subsection{Design and Implementation Constraints}
\begin{itemize}
    \item Free-tier limitations on Google Colab (GPU session timeouts)
    \item Data privacy and interstellar regulation compliance
\end{itemize}

\subsection{User Documentation}
\begin{itemize}
    \item README file in GitHub repository
    \item Markdown or PDF user guide on how to run the model
\end{itemize}

\subsection{Assumptions and Dependencies}
\begin{itemize}
    \item The Kaggle data is assumed to be representative and labeled accurately.
    \item Internet access is available for Colab usage.
\end{itemize}

\section{Specific Requirements}
\subsection{Functional Requirements}
\begin{itemize}
    \item \textbf{FR1:} System must load a CSV dataset containing passenger data.
    \item \textbf{FR2:} System must clean and preprocess data for missing or inconsistent entries.
    \item \textbf{FR3:} System must train a classification model.
    \item \textbf{FR4:} System must output predictions on whether a passenger was transported.
    \item \textbf{FR5:} System must export predictions to a CSV file.
\end{itemize}

\subsection{Non-Functional Requirements}
\begin{itemize}
    \item \textbf{NFR1:} Predictions must complete within 60 seconds for datasets under 1,000 rows.
    \item \textbf{NFR2:} Model accuracy must exceed 80\% AUC on validation data.
    \item \textbf{NFR3:} Data must remain secure and anonymized throughout processing.
    \item \textbf{NFR4:} Code must follow PEP8 style guidelines.
\end{itemize}

\subsection{External Interface Requirements}
\begin{itemize}
    \item \textbf{User Interface:} Simple command-line or web interface to input files and display results.
    \item \textbf{Hardware Interfaces:} None (cloud-based).
    \item \textbf{Software Interfaces:} Google Colab, Python 3.x, ML libraries.
\end{itemize}

\section{Appendices}
\subsection{Appendix A: Future Enhancements}
\begin{itemize}
    \item Integrate with a web app for live input/output.
    \item Add model explainability features (e.g., SHAP values).
    \item Automate retraining when new data is added.
\end{itemize}

\subsection{Appendix B: Revision History}
\begin{tabular}{|l|l|l|l|}
\hline
Version & Date & Author & Description \\
\hline
1.0 & \today & Gauri Chahal & Initial SRS draft \\
\hline
\end{tabular}

\end{document}
