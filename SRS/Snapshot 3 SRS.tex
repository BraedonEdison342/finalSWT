\documentclass[15pt]{article}
\usepackage[a4paper, margin=1in]{geometry}
\usepackage{titlesec}
\usepackage{hyperref}
\usepackage{graphicx}
\usepackage{fancyhdr}
\usepackage{enumitem}

\pagestyle{fancy}
\fancyhf{}
\rhead{Software Requirements Specification}
\lhead{Titanic Space Ship AI Model}
\rfoot{\thepage}

\title{Software Requirements Specification (SRS)\\\large Spaceship Titanic AI Prediction System}
\author{Team Name or Author(s)}
\date{\today}

\begin{document}

\maketitle
\tableofcontents
\newpage

\section{Introduction}
\subsection{Purpose}
This document defines the functional and non-functional requirements for a machine learning–based system designed to predict whether a passenger aboard the Spaceship Titanic was transported to another dimension due to a spacetime anomaly.

\subsection{Scope}
The system will ingest spaceship passenger data, preprocess it, train a model using historical outcomes, generate predictions for new data, provide explainability, and present results on an interactive dashboard.

\subsection{Definitions, Acronyms, and Abbreviations}
\begin{table}[h!]
  \centering
  \begin{tabular}{|l|l|p{8cm}|}
  \hline
  \textbf{Acronym} & \textbf{Full Form}                       & \textbf{Description} \\
  \hline
  AI    & Artificial Intelligence                  & Systems that simulate human intelligence in machines. \\
  ML    & Machine Learning                         & A subset of AI that enables models to learn from data. \\
  CSV   & Comma-Separated Values                   & Common format for structured tabular data files. \\
  CFG   & Configuration File                       & A JSON or YAML file specifying runtime parameters. \\
  SHAP  & SHapley Additive exPlanations            & A method for explainable AI to attribute feature importance. \\
  DASH  & Dash by Plotly                           & Framework for building interactive dashboards in Python. \\
  API   & Application Programming Interface        & A set of rules allowing software components to communicate. \\
  SRS   & Software Requirements Specification      & A document that describes what the system should do. \\
  \hline
  \end{tabular}
  \caption{Acronyms and Abbreviations}
\end{table}

\subsection{References}
\begin{itemize}
  \item Kaggle Competition: \url{https://www.kaggle.com/competitions/spaceship-titanic/overview}
  \item IEEE Standard for SRS: IEEE 830
  \item Python Library Docs: \texttt{pandas}, \texttt{scikit-learn}, \texttt{xgboost}, \texttt{shap}, \texttt{dash}
\end{itemize}

\subsection{Overview}
This document outlines detailed requirements: system features, interfaces, constraints, and acceptance criteria, now including explainability and dashboard capabilities.

\section{Overall Description}
\subsection{Product Perspective}
Standalone prediction system augmented with model explanability and interactive dashboard features.

\subsection{Product Functions}
\begin{itemize}
  \item Load passenger data (CSV).
  \item Preprocess data.
  \item Train ML model.
  \item Predict transported status.
  \item Read runtime parameters from configuration file.
  \item Log errors and send alerts.
  \item Compute SHAP explainability reports (global & local).
  \item Display results on an interactive Dash dashboard.
\end{itemize}

\subsection{User Classes and Characteristics}
\begin{itemize}
  \item \textbf{Data Scientists:} Maintain pipeline, interpret SHAP outputs.
  \item \textbf{Rescue Analysts:} Use dashboard to prioritize rescue.
  \item \textbf{Developers:} Extend system; add new explainability methods.
\end{itemize}

\subsection{Operating Environment}
\begin{itemize}
  \item Google Colab / Jupyter Notebook
  \item Python 3.x
  \item Modern browser (Chrome, Firefox) for Dash
\end{itemize}

\subsection{Design and Implementation Constraints}
\begin{itemize}
  \item Colab GPU/TPU session limits.
  \item Data privacy and interstellar regulation compliance.
  \item Dashboard performance: must paginate and cache large datasets.
\end{itemize}

\subsection{User Documentation}
\begin{itemize}
  \item README + extended user guide for explainability and dashboard usage.
\end{itemize}

\subsection{Assumptions and Dependencies}
\begin{itemize}
  \item Representative labeled data.
  \item Working `config.json`.
  \item Access to Slack/email for alerts.
\end{itemize}

\section{Specific Requirements}
\subsection{Functional Requirements}
\begin{itemize}
  \item \textbf{FR1–FR7:} As in Snapshot 2.
  \item \textbf{FR8:} System must compute SHAP global summary and local explanations for selected records.
  \item \textbf{FR9:} System must host an interactive Dash dashboard to filter, visualize predictions and display SHAP plots.
  \item \textbf{FR10:} Dashboard must allow exporting the current view to CSV or PNG.
\end{itemize}

\subsection{Non-Functional Requirements}
\begin{itemize}
  \item \textbf{NFR1–NFR5:} As in Snapshot 2.
  \item \textbf{NFR6:} SHAP report generation must complete within 120 seconds for up to 5,000 records.
  \item \textbf{NFR7:} Dashboard interactions (filter, threshold slide) must respond within 3 seconds.
  \item \textbf{NFR8:} All code must adhere to PEP8 and dashboard UI styles.
\end{itemize}

\subsection{External Interface Requirements}
\begin{itemize}
  \item \textbf{User Interface:} Dash app at a known URL; CLI fallback.
  \item \textbf{Software Interfaces:} Python 3.x, ML and SHAP libraries, Dash.
\end{itemize}

\section{Appendices}
\subsection{Appendix A: Future Enhancements}
\begin{itemize}
  \item Add Docker-compose definitions for full stack.
  \item Integrate real-time data streaming.
  \item Implement alternative explainability (LIME).
\end{itemize}

\subsection{Appendix B: Revision History}
\begin{tabular}{|l|l|l|l|}
\hline
Version & Date & Author & Description \\
\hline
1.0 & \today & Gauri Chahal & Initial SRS draft \\
\hline
1.1 & \today & Gauri Chahal & Added configuration and logging requirements \\
\hline
1.2 & \today & Gauri Chahal & Added SHAP explainability and dashboard requirements \\
\hline
\end{tabular}

\end{document}
